\documentclass[runningheads]{llncs}

%---- Sonderzeichen-------%
\usepackage[ngerman]{babel}
%---- Codierung----%
\usepackage[utf8]{inputenc}
%\usepackage[latin1]{inputenc}
\usepackage[T1]{fontenc}
\usepackage{graphicx}
\usepackage{url}
\usepackage{llncsdoc}
%----- Mathematischer Zeichenvorrat---%
\usepackage{amsmath}
\usepackage{amssymb}
\usepackage{enumerate}
% fuer die aktuelle Zeit
\usepackage{scrtime}
\usepackage{listings}
\usepackage{subfigure}
\usepackage{hyperref}

\setcounter{tocdepth}{3}
\setcounter{secnumdepth}{3}

% -------------------------------------------------------------------------------------------------
% -------------------------------------------------------------------------------------------------

\mainmatter
\title{Universal Rendering}
\titlerunning{Universal Rendering}
\author{Julian Beck}
\authorrunning{Julian Beck}
\institute{Betreuer: Prof. Dr. rer. nat. Christian Zirpins}
\date{01.05.2019}
\begin{document}
\let\oldaddcontentsline\addcontentsline
\def\addcontentsline#1#2#3{}
\maketitle
\def\addcontentsline#1#2#3{\oldaddcontentsline{#1}{#2}{#3}}

% -------------------------------------------------------------------------------------------------

\begin{abstract}
  An dieser Stelle sollte später eine Kurzzusammenfassung stehen.
\end{abstract}

% -------------------------------------------------------------------------------------------------
\tableofcontents 
\newpage
% -------------------------------------------------------------------------------------------------

\section{Einleitung}
\label{sec:Einleitung}

Seit dem Beginn des Webs funktioniert das Surfen wie folgt: Ein Webbrowser fordert eine bestimmte Seite an, ein Server im Internet bearbeitet die Anfrage und generiert ein HTML (Hypertext Markup Language) Dokument als Antwort. Dies bezeichnet man als serverseitiges rendern. In den Anfängen des Webs stellte dies kein Problem dar, da die Browser nicht leistungsstark waren und die Webseiten aus meist statischen Inhalt bestanden. Später mit HTML5 wurden Webseiten dynamischer und interaktiver für den Nutzer, was dazu führte, dass immer mehr Apps, sogenannte Single Page Applikationen, vollständig im Browser auf einer Seite liefen. Um dies zu ermöglichen wird clientseitiges rendern verwendet. Single Page Applications oder kurz SPAs, bieten Vorteile für den Anwender: Sie reagieren schnell auf Benutzerinteraktionen und können zwischen Seiten navigieren, ohne sie komplett neu zu laden. Gleichzeitig sind SPAs komfortabel zu entwickeln, dank moderner Frameworks. Beide Varianten, serverseitiges und clientseitiges rendern, haben Vor- und Nachteile. Universal Rendering kombiniert die beiden Ansätze und erfüllt alle Anforderungen an eine moderne Webanwendung.


\subsection{Anforderungen an eine Webanwendung}
\label{subsec:Anforderungen an eine Webanwendung}

\subsection{Terminologie}
\label{subsec:Terminologie}

% -------------------------------------------------------------------------------------------------

\section{Serverseitiges Rendering}
\label{sec:Serverseitiges Rendering}

\begin{figure}[h]
  \centering
  \includegraphics[width=12cm]{images/server}
  \caption{HTMl Dokument einer React Seite}
\end{figure}



\subsection{Serverseitiges Rendering mit Ajax}
\label{subsec:Serverseitiges Rendering mit Ajax}

\begin{figure}[h]
  \centering
  \includegraphics[width=12cm]{images/serverajax}
  \caption{HTMl Dokument einer React Seite}
\end{figure}


Dies ist ein Zitat \cite{becker2008a}.
test\cite{IsomorphicApps}
test\cite{SearchFriendly}
test\cite{chen_chen_2016}

\newpage
% -------------------------------------------------------------------------------------------------

\section{Clientseitiges Rendering}
\label{sec:Clientseitiges Rendering}

\begin{figure}[h]
  \centering
  \includegraphics[width=12cm]{images/client}
  \caption{HTMl Dokument einer React Seite}
\end{figure}

\begin{figure}[h]
  \centering
  \includegraphics[width=9cm]{images/react-code-small}
  \caption{HTMl Dokument einer React Seite}
\end{figure}


\newpage
% -------------------------------------------------------------------------------------------------

\section{Universal Rendering}
\label{sec:Universal Rendering}

\begin{figure}[h]
  \centering
  \includegraphics[width=12cm]{images/react}
  \caption{HTMl Dokument einer React Seite}
\end{figure}


\begin{figure}[h]
  \centering
  \includegraphics[width=10cm]{images/universalseo}
  \caption{HTMl Dokument einer React Seite}
\end{figure}

\subsection{Isomorphic JavaScript}
\label{subsec:Isomorphic JavaScript}

\subsection{Virtuelles DOM}
\label{subsec:Virtuelles DOM}

\begin{figure}[h]
  \centering
  \includegraphics[width=5cm]{images/caching}
  \caption{HTMl Dokument einer React Seite}
\end{figure}

\subsection{Clientseitige Hydration}
\label{subsec:Clientseitige Hydration}

\subsection{Rendering Ablauf}
\label{subsec:Rendering Ablauf}

\begin{figure}[h]
  \centering
  \includegraphics[width=12cm]{images/universal}
  \caption{HTMl Dokument einer React Seite}
\end{figure}

\newpage
\subsection{Vorteile}
\label{subsec:Vorteile}

\begin{figure}[h]
  \centering
  \includegraphics[width=9cm]{images/nuxt-body-first}
  \caption{HTMl Dokument einer React Seite}
\end{figure}

\subsection{Nachteile}
\label{subsec:Nachteile}

\subsection{Alternativen}
\label{subsec:Alternativen}

\newpage
% -------------------------------------------------------------------------------------------------

\section{Frameworks}
\label{sec:Evaluation}

\subsection{React und Next.js}
\label{subsec:React und Next.js}

\begin{figure}
  \centering
  \includegraphics[width=10cm]{images/CodeSplitting}
  \caption{HTMl Dokument einer React Seite}
\end{figure}

\begin{figure}
  \centering
  \includegraphics[width=7cm]{images/prefetchnext}
  \caption{HTMl Dokument einer React Seite}
\end{figure}


\subsection{Vue.js und Nuxt.js}
\label{subsec:Vue.js und Nuxt.js}

\subsection{Angular Universal}
\label{subsec:Angular Universal}

\newpage
% -------------------------------------------------------------------------------------------------

\section{Universal Rendering in der Praxis}
\label{sec:Universal Rendering in der Praxis}

\newpage
% -------------------------------------------------------------------------------------------------

\section{Fazit und Ausblick}
\label{sec:Fazit}

\begin{figure}
  \centering
  \includegraphics[width=10cm]{images/HackerNews}
  \caption{HTMl Dokument einer React Seite}
\end{figure}

\begin{figure}
  \centering
  \includegraphics[width=10cm]{images/JavaScriptGoogleShips}
  \caption{HTMl Dokument einer React Seite}
\end{figure}




% -------------------------------------------------------------------------------------------------
\newpage
% Normaler LNCS Zitierstil
%\bibliographystyle{splncs}
\bibliographystyle{itmalpha}
% TODO: Ändern der folgenden Zeile, damit die .bib-Datei gefunden wird
\bibliography{literatur}

\end{document}
