\documentclass[runningheads]{llncs}

%---- Sonderzeichen-------%
\usepackage[ngerman]{babel}
%---- Codierung----%
\usepackage[utf8]{inputenc}
%\usepackage[latin1]{inputenc}
\usepackage[T1]{fontenc}
\usepackage{graphicx}
\usepackage{url}
\usepackage{llncsdoc}
%----- Mathematischer Zeichenvorrat---%
\usepackage{amsmath}
\usepackage{amssymb}
\usepackage{enumerate}
% fuer die aktuelle Zeit
\usepackage{scrtime}
\usepackage{listings}
\usepackage{subfigure}
\usepackage{hyperref}

\setcounter{tocdepth}{3}
\setcounter{secnumdepth}{3}

% -------------------------------------------------------------------------------------------------
% -------------------------------------------------------------------------------------------------

\mainmatter
\title{Server Side Rendering}
\titlerunning{Server Side Rendering}
\author{Julian Beck}
\authorrunning{Julian Beck}
\institute{Betreuer: Prof. Dr. rer. nat. Christian Zirpins}
\date{01.05.2019}
\begin{document}
\let\oldaddcontentsline\addcontentsline
\def\addcontentsline#1#2#3{}
\maketitle
\def\addcontentsline#1#2#3{\oldaddcontentsline{#1}{#2}{#3}}

% -------------------------------------------------------------------------------------------------

\begin{abstract}
  An dieser Stelle sollte später eine Kurzzusammenfassung stehen.
\end{abstract}

% -------------------------------------------------------------------------------------------------
\tableofcontents 
\newpage

\section{Einleitung}
\label{sec:Einleitung}


% -------------------------------------------------------------------------------------------------

\section{Grundlagen}
\label{sec:Grundlagen}


Dies ist ein Zitat \cite{becker2008a}.

\subsection{Beispiel: Tabellen}
\label{subsec:Tables}


\begin{table}[h]
  \centering
  \caption{Eine Tabelle}
  \label{tab:atable}
  \begin{tabular}{|r|l|} \hline
    abc & def\\ \hline
    ghi & jkl\\ \hline
    123 & 456\\ \hline
    789 & 0AB\\ \hline
  \end{tabular}
\end{table}


\subsection{Beispiel: Abbildungen}
\label{subsec:Abbildungen}


\begin{figure}[h]
  \centering
  \includegraphics[width=4cm]{IWI-HsKA_CMYK_V01}
  \caption{IWI Logo}
  \label{fig:iwilogo}
\end{figure}


% -------------------------------------------------------------------------------------------------

\section{Konzeption}
\label{sec:Konzeption}

% -------------------------------------------------------------------------------------------------

\section{Durchführung}
\label{sec:Durchführung}

% -------------------------------------------------------------------------------------------------

\section{Evaluation und Diskussion}
\label{sec:Evaluation}


% -------------------------------------------------------------------------------------------------

\section{Fazit und Ausblick}
\label{sec:Fazit}


% -------------------------------------------------------------------------------------------------

% Normaler LNCS Zitierstil
%\bibliographystyle{splncs}
\bibliographystyle{itmalpha}
% TODO: Ändern der folgenden Zeile, damit die .bib-Datei gefunden wird
\bibliography{literatur}

\end{document}
