\documentclass[runningheads]{llncs}

%---- Sonderzeichen-------%
\usepackage[ngerman]{babel}
%---- Codierung----%
\usepackage[utf8]{inputenc}
%\usepackage[latin1]{inputenc}
\usepackage[T1]{fontenc}
\usepackage{graphicx}
\usepackage{url}
\usepackage{llncsdoc}
%----- Mathematischer Zeichenvorrat---%
\usepackage{amsmath}
\usepackage{amssymb}
\usepackage{enumerate}
% fuer die aktuelle Zeit
\usepackage{scrtime}
\usepackage{listings}
\usepackage{subfigure}
\usepackage{hyperref}
\usepackage{lipsum}

\setcounter{tocdepth}{3}
\setcounter{secnumdepth}{3}

% -------------------------------------------------------------------------------------------------
% -------------------------------------------------------------------------------------------------
\begin{document}

\mainmatter
\title{Titel der Seminararbeit}
\titlerunning{Titel der Seminararbeit}
\author{Name des Autors}
\authorrunning{Titel des Seminars}
\institute{Betreuer: Betreuername}
\date{23.07.2007}
\maketitle

% -------------------------------------------------------------------------------------------------

\begin{abstract}
  An dieser Stelle sollte später eine Kurzzusammenfassung stehen.
\end{abstract}

% -------------------------------------------------------------------------------------------------

\section{Einleitung}
\label{sec:Einleitung}

\lipsum

% -------------------------------------------------------------------------------------------------

\section{Grundlagen}
\label{sec:Grundlagen}

\lipsum[1-2]

Dies ist ein Zitat \cite{becker2008a}.

\subsection{Beispiel: Tabellen}
\label{subsec:Tables}

\lipsum[3]

\begin{table}[h]
  \centering
  \caption{Eine Tabelle}
  \label{tab:atable}
  \begin{tabular}{|r|l|} \hline
    abc & def\\ \hline
    ghi & jkl\\ \hline
    123 & 456\\ \hline
    789 & 0AB\\ \hline
  \end{tabular}
\end{table}

\lipsum[4]

\subsection{Beispiel: Abbildungen}
\label{subsec:Abbildungen}

\lipsum[5-6]

\begin{figure}[h]
  \centering
  \includegraphics[width=4cm]{IWI-HsKA_CMYK_V01}
  \caption{IWI Logo}
  \label{fig:iwilogo}
\end{figure}

\lipsum[7]

% -------------------------------------------------------------------------------------------------

\section{Konzeption}
\label{sec:Konzeption}
\lipsum[8-10]

% -------------------------------------------------------------------------------------------------

\section{Durchführung}
\label{sec:Durchführung}
\lipsum[11-13]

% -------------------------------------------------------------------------------------------------

\section{Evaluation und Diskussion}
\label{sec:Evaluation}

\lipsum[14-15]

% -------------------------------------------------------------------------------------------------

\section{Fazit und Ausblick}
\label{sec:Fazit}

\lipsum[16]

% -------------------------------------------------------------------------------------------------

% Normaler LNCS Zitierstil
%\bibliographystyle{splncs}
\bibliographystyle{itmalpha}
% TODO: Ändern der folgenden Zeile, damit die .bib-Datei gefunden wird
\bibliography{literatur}

\end{document}
