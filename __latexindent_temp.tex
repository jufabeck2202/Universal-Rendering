\documentclass[runningheads]{llncs}

%---- Sonderzeichen-------%
\usepackage[ngerman]{babel}
%---- Codierung----%
\usepackage[utf8]{inputenc}
%\usepackage[latin1]{inputenc}
\usepackage[T1]{fontenc}
\usepackage{graphicx}
\usepackage{url}
\usepackage{llncsdoc}
%----- Mathematischer Zeichenvorrat---%
\usepackage{amsmath}
\usepackage{amssymb}
\usepackage{enumerate}
% fuer die aktuelle Zeit
\usepackage{scrtime}
\usepackage{listings}
\usepackage{subfigure}
\usepackage{hyperref}

\setcounter{tocdepth}{3}
\setcounter{secnumdepth}{3}

% -------------------------------------------------------------------------------------------------
% -------------------------------------------------------------------------------------------------

\mainmatter
\title{Server Side Rendering}
\titlerunning{Server Side Rendering}
\author{Julian Beck}
\authorrunning{Julian Beck}
\institute{Betreuer: Prof. Dr. rer. nat. Christian Zirpins}
\date{01.05.2019}
\begin{document}
\let\oldaddcontentsline\addcontentsline
\def\addcontentsline#1#2#3{}
\maketitle
\def\addcontentsline#1#2#3{\oldaddcontentsline{#1}{#2}{#3}}

% -------------------------------------------------------------------------------------------------

\begin{abstract}
  An dieser Stelle sollte später eine Kurzzusammenfassung stehen.
\end{abstract}

% -------------------------------------------------------------------------------------------------
\tableofcontents 
\newpage

\section{Einleitung}
\label{sec:Einleitung}
Seit dem Beginn des Webs funktioniert das Surfen wie folgt: Ein Webbrowser fordert eine bestimmte Seite an , wodurch ein Server im Internet mit einer HTML seite antwortet. In den Anfängen des Webs stellte dies kein Problem dar, da die Browser nicht leistungsstark waren und die Webseiten aus meist Statischen Seiten bestanden. Später mit der HMTL5 wurden Webseiten dynamischer und Interaktiver, was dazu führte das immer mehr Apps, sogenannte Single-Page-apps komplett im Browser liefen, ohne eine Anfrage an den Server zu schicken. SPA bieten Vorteile für den Nutzer, sie reagieren schnell auf Benutzer interaktionen und können zwischen Seiten Navigieren ohne sie komplett neu zu laden.


% -------------------------------------------------------------------------------------------------

\section{Klassische Websiten}
\label{sec:Klassische Websiten}

\subsection{Statische Websiten}
\label{subsec:Statische Websiten}

\subsection{Dynamische Websiten mit Ajax}
\label{subsec:Dynamische Websiten mit Ajax}



Dies ist ein Zitat \cite{becker2008a}.

% -------------------------------------------------------------------------------------------------

\section{Single Page Applications}
\label{sec:Single Page Applications}

% -------------------------------------------------------------------------------------------------

\section{Server Site Rendering}
\label{sec:Server Site Rendering}

\subsection{Isomorphic JavaScript}
\label{subsec:Isomorphic JavaScript}

\subsection{Virtual DOM}
\label{subsec:Virtual DOM}

\subsection{Performance}
\label{subsec:Virtual DOM}
% -------------------------------------------------------------------------------------------------

\section{Frameworks}
\label{sec:Evaluation}


% -------------------------------------------------------------------------------------------------

\section{Fazit und Ausblick}
\label{sec:Fazit}


% -------------------------------------------------------------------------------------------------

% Normaler LNCS Zitierstil
%\bibliographystyle{splncs}
\bibliographystyle{itmalpha}
% TODO: Ändern der folgenden Zeile, damit die .bib-Datei gefunden wird
\newpage
\bibliography{literatur}

\end{document}
